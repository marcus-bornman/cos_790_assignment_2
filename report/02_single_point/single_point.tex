\section{Single-Point Search}\label{sec:single_point}
The single-point search selection perturbative hyper-heuristic applies a low-level heuristic - from the heuristic set in table \ref{tab:low-level_heuristics} - \emph{n} times, where \emph{n} has been arbitrarily chosen as the number of exams that need to be scheduled for the problem instance.

To select the low-level heuristic to apply a greedy selection technique is used. So, for each of the \emph{n} iterations, all low-level heuristics are applied to the existing solution to produce 5 new solutions - one for each low-level heuristic. Then, out of these 5 solutions, the solution which violates the least hard constraints is chosen to replace the current solution. Algorithm \ref{alg:single_point} details this single-point search approach in its entirety.

\begin{algorithm}[H]\label{alg:single_point}
\SetAlgoLined
 currentSolution = initialSolution\;
 \BlankLine
 \For{$i\gets1$ \KwTo $n$}{
 \BlankLine
   \ForEach{$heuristic$ in $H$}{
       newSolution = heuristic.applyTo(currentSolution)\;
       \uIf{newSolution.violations() $<$ bestSolution.violations()}{
            bestSolution = newSolution\;
       }
     }
     currentSolution = bestSolution\;
     \BlankLine
 }
 \BlankLine
 return currentSolution\;
 \caption{Single-Point Search Selection Perturbative Hyper-Heuristic}
\end{algorithm}