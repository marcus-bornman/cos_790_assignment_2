\section{Multi-Point Search}\label{sec:multi_point}
As has been mentioned, the multi-point selection perturbative hyper-heuristic employs a genetic algorithm which explores the space of heuristic combinations. This approach is similar to that of Raghavjee and Pillay \cite{raghavjee2015genetic} in that low-level heuristic combinations are evolved to produce the combinations that best improve on an initial solution. In addition, Evohyp \cite{pillay2017evohyp} - a Java toolkit for evolutionary algorithm hyper-heuristics - was used for the implementation of the genetic algorithm.

A single chromosome in the genetic algorithm population is represented as a heuristic combination - for example, \emph{AEBDACA}. To evaluate the chromosome, the heuristic combination is applied to the initial solution to produce a new solution. Each low-level heuristic is applied to the solution until the solution has been changed \emph{n} times, where \emph{n} has again been arbitrarily chosen as the number of exams that need to be scheduled for the problem instance. The means of applying a perturbative heuristic sequence is shown in algorithm \ref{alg:multi_point}. 

Following the evaluation a heuristic sequence, the fitness of a solution is calculated in the same manner as suggested by Pillay \cite{pillay2010evolving} in that fitness is a measure of the soft constraint cost, multiplied by the hard constraint cost incremented by one.

\begin{algorithm}[H]\label{alg:multi_point}
\SetAlgoLined
 currentSolution = initialSolution\;
 \BlankLine
 \For{$i\gets1$ \KwTo $n$}{
    heuristic = H.elementAt($i \mod H.length$)\;
    currentSolution = heuristic.applyTo(currentSolution)\;
 }
 \BlankLine
 return currentSolution\;
 \caption{Applying a Perturbative Heuristic Sequence}
\end{algorithm}

Furthermore, the genetic algorithm employs tournament selection as the selection method and uses crossover and mutation as genetic operators - these techniques are all supported by the Evohyp \cite{pillay2017evohyp} toolkit. Of course, as is the case with any genetic algorithm approach, there are specific parameters required by the algorithm. The parameters, in this case, were largely chosen based on proportions that have been used previously with similar approaches to the same benchmark set \cite{pillay2010evolving, raghavjee2015genetic}. However, many parameters were scaled down to smaller values. The genetic parameters used were selected as follows:

\begin{itemize}
    \item The Population Size was selected as 50.
    \item The Tournament Size was selected as 5.
    \item The Number of Generations was selected as 10.
    \item The Mutation Rate was selected as 0.3.
    \item The Crossover rate was selected as 0.7.
    \item The Maximum Initial Length was selected as 20.
    \item The Maximum Offspring Length was selected as 20.
    \item The Mutation Length was selected as 5.
\end{itemize}